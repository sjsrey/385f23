
\documentclass{article} %this needs to be at the top of each document

% margins, size, formatting 
\oddsidemargin=.25in
\evensidemargin=.25in
\topmargin=-.5in
\textwidth=6in
\textheight=9in
\pagestyle{empty}

\setlength{\parindent}{0in} %turn off paragraph indentation

% packages for fancy fonts, symbols, theoremm/proof environments, etc - go ahead and use this in your template, whether you need it or not!
\usepackage{amsmath,amssymb,amsthm}
\usepackage{graphicx}
\usepackage[pdftex,
pdftitle={GPH 483/598},
pdfauthor={Sergio J. Rey},
pdfkeywords={ESDA},
pdfpagemode=UseOutlines,
bookmarks,bookmarksopen,pdfstartview=FitH,
colorlinks,linkcolor=blue,citecolor=blue,urlcolor=blue,
]
{hyperref}

\begin{document}


{\large \bf GPH 483/598 Geographic Information Analysis \hfill Spring 2010\\
\hfill Exercise 1}
\vspace{.2in}
\hrule 
\vspace{.1in} 
\textbf{Instructions:} Using \texttt{R} you are to carry out a statistical
analysis of the problem below. Your work must be turned in on the blackboard
site and include: [1] Your script; [2] Your short interpretation and findings.

Due: February 10, 13:00 mst. 
\vspace{.1in}
\hrule 
\vspace{.2in}
{\bf Research Question:} You are an economic geographer interested in the
impact of accessibility on residential rental markets.
More specifically, you want to investigate the hypothesis
that apartment rents per square foot are significantly higher for locations
within 1,000 feet  of a trolley stop, compared to apartments  that
are further than 1,000 feet away from the nearest trolley stop.

\vspace{.2in}

{\bf Data:} The file \texttt{trolley.dat} has the sample data for your
analysis. It has $N$ rows, with the first row being a header. Each subsequent
row is either an apartment  or a trolley record. If the record is for an
apartment, the value in the first column is the
total monthly rent, $R$ for the apartment, the second column is the square
footage of the apartment $s$, while the third and forth column are the $x$ and
$y$ locational coordinates for the apartment. If the record is for a trolley
station, the values are -9 for the first two columns, while the third
and forth columns are the locational coordinates for the trolley station.

\vspace{.2in}

All jobs in this city are in the central business district (CBD), and all
commuters must go to one of the trolley stations to commute to the CBD each
day which is located at $(x=0,y=0)$.  You are interested in testing if the
apartments within 1,000 feet of a trolley station command a
significantly higher rent (per sq foot) than those beyond 1,000 feet of a
station.

\vspace{.2in}
Suppose you also know that the apartment rent gradient (rent per square foot
at different distances) is as follows:
\begin{equation}
  r_{i}=r_0-0.015 (d_i) + \beta T_i
  \label{e:r}
\end{equation}
where $r_{i}$ is the rent (in cents) per square foot of apartment space at
$d_i$ feet from the CBD, $r_0$ is the rent at the CBD, $\beta$  is the amount by which apartment rents per
square foot increase if the apartment is within 1,000 feet of a trolley
station, and $T_i=1$ if $i$ is within 1,000 feet of a trolley station
otherwise $T_i=0$. 

\vspace{.2in}
You also know that the apartment size gradient is:
\begin{equation}
  s_{i}=500+0.20 (d_i)
  \label{e:s}
\end{equation}
where $s_i$ is the square footage of an apartment at $d_i$ feet from the
CBD. Finally, the total monthly rent for an apartment is given as:

\begin{equation}
  R_i = s_i \times r_i \ .
  \label{e:R} 
\end{equation}



\vspace{.2in}

{\bf Tasks:} 
\begin{enumerate}
  \item Process your data:
    \begin{enumerate}
      \item What is the value of $N$?
      \item Determine how many trolley stations are in the sample
      \item Determine how many apartments are in the sample
      \item What is the size of the bounding rectangle that contains all the
      apartments?
    \end{enumerate}
  \item Visualize your data:
  \begin{enumerate}
    \item Plot the locations of the apartments
    \item On the same plot, add the locations of the trolley stations and a 1,000 foot circular
      buffer around each station
    \item On a separate graph, plot the size gradient, $s$
    \item Plot the total monthly rent gradient, $R$
    \item Plot the rent per square foot gradient, $r$
    \item Provide an interpretation of each of your plots
  \end{enumerate}
  \item Computationally, determine which apartments are within 1,000 feet of a trolley stop and
    which are not (assume distance can be measured as the crow flies).
  \item What is the value of $r_0$?
  \item Clearly formulate your null and alternative hypotheses regarding your
    research question
  \item Based on your sample, provide an estimate of $\beta$
  \item Select a test statistic
  \item Calculate the statistic and its p-value
  \item Interpret your findings with regard to the hypotheses
\end{enumerate}


\end{document} % every document must end with this.

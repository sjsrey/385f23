
\documentclass{article} %this needs to be at the top of each document

% margins, size, formatting 
\oddsidemargin=.25in
\evensidemargin=.25in
\topmargin=-.5in
\textwidth=6in
\textheight=9in
\pagestyle{empty}

\setlength{\parindent}{0in} %turn off paragraph indentation

% packages for fancy fonts, symbols, theoremm/proof environments, etc - go ahead and use this in your template, whether you need it or not!
\usepackage{amsmath,amssymb,amsthm}
\usepackage{graphicx}
\usepackage{multirow}
\usepackage[pdftex,
pdftitle={GPH 483/598},
pdfauthor={Sergio J. Rey},
pdfkeywords={ESDA},
pdfpagemode=UseOutlines,
bookmarks,bookmarksopen,pdfstartview=FitH,
colorlinks,linkcolor=blue,citecolor=blue,urlcolor=blue,
]
{hyperref}

\begin{document}


{\large \bf GPH 483/598 Geographic Information Analysis \hfill Spring 2010}
\vspace{.2in} %This command leaves blank vertical space of the amount described.  I usually use inches.  This is useful if you need to leave space to draw a diagram.  It does not work well at the end of a page.
\hrule 
\vspace{.1in} %This command leaves blank vertical space of the amount described.  I usually use inches.  This is useful if you need to leave space to draw a diagram.  It does not work well at the end of a page.
\textbf{Instructor:} Dr. Sergio J. Rey\\
\textbf{e-mail:} \texttt{srey@asu.edu}\\
\textbf{Phone:} (619) 928-4499\\
\textbf{URL:} \texttt{www.geoplan.asu/rey}\\
\textbf{Office:} Coor 512\\
\textbf{Office Hours:} Wed 1-2 and by appointment
\vspace{.1in} %This command leaves blank vertical space of the amount described.  I usually use inches.  This is useful if you need to leave space to draw a diagram.  It does not work well at the end of a page.
\hrule 
\vspace{.2in}
{\bf COURSE DESCRIPTION:} The purpose of this course is to introduce you to
advanced methods of exploratory spatial data analysis (ESDA).  The focus is on
both conceptual and applied aspects of ESDA methods. We will place particular
emphasis on the computational aspects of ESDA methods for three different
types of spatial data: point processes, lattice, and geostatistical. Throughout
the course you will gain valuable hands-on experience with several specialized
software packages for ESDA.  The overriding goal of the course is for you to
acquire familiarity with the fundamental methodological and operational issues
in ESDA and the ability to extend these methods in your own research.

\vspace{.2in}

{\bf PREREQUISITES:}
All participants are expected to have  working knowledge of spatial analysis
concepts and to be familiar with multivariate statistics. No extensive GIS
background beyond ArcGIS basics is needed.


\vspace{.2in}

{\bf READINGS:} \emph{Geographic Information Analysis}, David O'Sullivan and
David J Unwin (required).
Supplementary Texts: (not required):\\
\emph{Geospatial Analysis}, Michael De
Smith, Michael F. Goodchild, Paul A. Longley\\
\emph{Applied Spatial Data Analysis with R}, Roger S. Bivand, Edzer J.
Pebesma, and Virgilio G\'omez-Rubio.

\vspace{.2in}


{\bf GRADING:}  Grading will be assinged on the following scale: A: 90-100, B:
80-89, C:70-79, D: 60-69, F below 60. There will be no curves and no extra
credit. I will assign +/- on an individual basis. Points are assigned as
follows:

\vspace{.1in}
\begin{center}
\begin{tabular}[h]{|l|rr|}
  \hline
  &\multicolumn{2}{|c|}{Points}\\
  Component&Undergraduate&Graduate\\
  \hline
  Exercises&40&30\\
  Project&40&40\\
  Quizzes&20&40\\
  Presentations& &10\\
  \hline
  TOTAL&100&100\\ \hline
\end{tabular}
\end{center}
\vspace{.1in}

{\em Exercises:} A series of graded class assignments will be given for each
of the major topics (point patterns, lattice, geostatistics) which focus on
the mechanics of carrying out an analysis.  Your exercises will be introduced
in the second half of a class meeting and are to be completed on your own time
outside of class. You are welcome to collaborate with other students on these
exercise but the work that you hand in must be your own. 


\vspace{.1in}

{\em Project:} The final project will consist of a data analysis and  writeup
for 40\% of the grade. You will have to select a spatial data set and carry
out an in-depth analysis of spatial pattern. Your selection will need to be
approved before you can proceed with the project. Graduate students must find
their own data, while undergraduate students have the option of finding their
own data or having data assigned by me. Graduate students will also be
required to give a presentation of their project at the end of the course.


\vspace{.1in}

{\em Quizzes:} Starting the second week of the semester, a short quiz will be
given during the first half of the class meeting. These will be based on material from the
previous lectures and are graded on a pass-fail basis. Absence on the day of a
quiz results in a 0 for that quiz. One quiz will be dropped in determining
your final quiz average.

\vspace{.1in}

{\em Article Presentations:} Each graduate student will be assigned additional
readings on a particular type of spatial analysis and will present a synopsis
of these articles to the course.


\vspace{.2in}

{\bf ORGANIZATION:} The course will meet in CoorL 1-18. The first part of each
weekly meeting will focus on background and theoretical material related to
the particular type of spatial data analysis. Following a short break, the
attention will shift to introduce various software packages for the analysis
of spatial data.

\vspace{.1in}

{\em Classroom etiquette:} To ensure a productive learning environment, all
participants are expected to abide by the following rules:
\begin{itemize}
  \item Because we are meeting in a computer laboratory,
food is strictly forbidden in the class meetings.
\item Use of the classroom computers should only be in support of the course.
  Extracurricular use (i.e., browsing noncourse materials, using  social networks, checking
  email, chat sessions, etc) during class meetings is disruptive for your colleagues and
  disrespectful. Individuals violating this rule will be asked to leave for
  that course session.
\end{itemize}

\vspace{.2in}

{\bf SOFTWARE:}
While this course does not have a formal lab unit, in the sense that you would
get an additional credit unit, the course material lends itself rather nicely
to computationally based instruction. We will utilize a set of freely available
software packages\footnote{%
The majority of the packages recommended can be classified as open source software.
This means, among other things, you are free to use this software without having
to pay any licensing fee, so long as you respect the copyright of the author
who wrote the package. For more details on the concept of open source
see http://www.opensource.org .} that you can put on your own (or a
friend's) computer to work on the examples and exercises in support of
the course material. We will set aside part of the lectures to
demonstrate various capabilities of these software packages to support
spatial analysis. You will have additional opportunity to apply these methods
both in the individual exercises as well as in your  project.
We will primarily rely on  the  package   \href{http://www.r-project.org/}{R} for 
which is freely available and runs on a number of platforms, including
Linux, Unix, Mac OS X, and Windows (all flavors). It is a very powerful
mathematical programming language with many data analysis, graphical and
computational functions.

\vspace{.2in}

{\bf SCHEDULE:} 
The proposed topics and their sequence are as follows:
\begin{table}[htbp]
\begin{center}
\begin{tabular}{|lr|l|l|l|}
\hline
Month & \multicolumn{1}{l|}{Date} & Topic & Out & Due \\ \hline
Jan & 20 & Intro, Spatial Data &  &  \\ 
 & 27 & Representation of Spatial Structure & Exercise 1 &  \\ 
Feb & 3 & Point Pattern Basics &  &  \\ 
 & 10 & Centrography & Exercise 2 & Exercise 1 \\ 
 & 17 & Advanced Point Patterns &  & \\ 
 & 24 & WRSA &  & Exercise 2  \\ 
Mar & 3 & Lattice Data Basics & Exercise 3 &  \\ 
 & 10 & Spatial Weights &  &  \\ 
 & 17 & Spring Break &  &  \\ 
 & 24 & Global Autocorrelation & Exercise 4 & Exercise 3 \\ 
 & 31 & Local Autocorrelation &  &  \\ 
Apr & 7 & Space-Time Autocorrelation &  & Exercise 4 \\ 
 & 14 & Geostatistics Basics & Exercise 5 &  \\ 
 & 21 & Variography &  &  \\ 
 & 28 & Kriging &  & Exercise 5 \\ 
May & 5 &  &  & Projects \\ 
 & 12 &  &  & Presentations \\ \hline
\end{tabular}
\end{center}
\end{table}



\end{document} % every document must end with this.
